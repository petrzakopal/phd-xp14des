\documentclass[a4paper, twoside, 11pt]{article}
% It is needed to use this command for automatic compilation in VSCode
% !TEX program = lualatexmk

%% DOCUMENT, PREAMBLE AND MACROS DESIGNED FOR LuaLaTeX %%
\newcommand{\fbar}{\FloatBarrier} % barier for stopping figures from leaking to other subsections, use as \fbar


\usepackage{amsmath} % math package
\usepackage{amssymb} % for miscellaneous mathematical symbols, first usage was for tick symbol in math mode \checkmark
\usepackage{textcomp} % for miscellaneous symbols
\usepackage{graphicx} % enhanced support for graphics
\usepackage{cmap} % mapování znaků - vyhledávání v pdf
% \usepackage[czech]{babel} % CZ
\usepackage[english]{babel} % EN
\usepackage[utf8]{inputenc} % kódování 
\usepackage[T1]{fontenc} % kódování 
\usepackage{multirow} % Multirow table support
\usepackage{float} % Improves the interface for defining floating objects such as figures and tables
\usepackage{wasysym} % for various glyphs, symbols

\usepackage{setspace} % spacing
\onehalfspacing

\usepackage{hyperref}
\hypersetup{
    colorlinks=true, % pokud nechci definovat citecolor=black aby byly odkazy citací černé, tak dám colorlinks=false,%
    bookmarks=true,
    linkcolor=black,
    citecolor=black,
    urlcolor=black,
}

% when using LuaLaTex, defining Times Fonts from your system - it has to be named like this and inserted ttf file in the folder of your tex file
\usepackage{fontspec}
\selectlanguage{czech}
\setmainfont[Ligatures=TeX,BoldFont={Times New Roman Bold}] {Times New Roman}
                                
\setsansfont[Ligatures=TeX,BoldFont={* Bold}]{Times New Roman}
                                      
\setmonofont{CourierPrime-Regular}
 
%\usepackage[italic]{mathastext} % for text in math environment, better looking times then



% for CITATIONS URL to work, it is not needed when you are not using URL label
\usepackage{url}
\usepackage{csquotes}
\usepackage[style=iso-numeric, backend=biber, isbn=true, urldate=iso,seconds=true, date=terse, datezeros=true, language=czech]{biblatex}
\addbibresource{src/bib/zdroje.bib} % BIB resources to import
%\DeclareUrlCommand\url{\def\UrlLeft{<}\def\UrlRight{>} \urlstyle{tt}}
%\usepackage{biblatex}
% END for citations %

% changing bibliography font
\renewcommand*{\bibfont}{\fontspec{Times New Roman}}

\usepackage{comment} % For comments
\usepackage{pdfpages} % for pdf pages
\usepackage{enumerate} % For lists
\usepackage{enumitem} % For Custom Numbering Nested Lists
\setlist[enumerate]{label*=\arabic*.} % setting Number. numbering in lists
\usepackage{tikz} % For vector graphics
\usepackage{circuitikz} % For schemes
\usepackage{pgf} % Post script graphics for tikz

% pouze funguje v PDFLaTeX%
%\usepackage{tgtermes} % na times font, jiný nefunguje s vyhledáváním a copy%

\usepackage{placeins} % for \FloatBarrier command that blocks floating with htbp! go over \FloatBarrier
\usepackage{mathrsfs} % package for math symbols for Laplace, Z transform etc., usage \mathscr{Z}
\usepackage{upgreek} % for upgreek symbols, specified tau \uptau
\usepackage{physics} % for derivations \dd
\usepackage[list=true,listformat=simple]{subcaption}
\usepackage[figurename=Fig.,font=small,labelfont=it,textfont=it]{caption} %for renaming figures instead of renewcommand, small for 11pt default is 10pt as needed in word template
\usepackage[tablename=Tab.,font=small,labelfont=it,
            textfont=it]{caption} % for renaming tables instead of renewcommand
            

%% GLOSSARIES %%
% List of abbreviations and symbols
% Original code author: Jakub Kučera

\usepackage[nonumberlist,nopostdot,section=subsection,numberedsection]{glossaries}
% section = subsection is for glossaries title to appear as a subsection, numberedsection adds the subsec number

\newglossary[slg]{symbolslist}{symbol}{ntn1}{List of symbols}
\newglossary[slg]{abbreviationslist}{abbreviation}{ntn2}{List of abbreviations}

\makeglossaries

% include files with definitions
% PZ definitions
\newglossaryentry{abbreviation:asm}{
                type=abbreviationslist,
                name={ASM},
                description={Asynchronní Motor}
}
\newglossaryentry{abbreviation:pmsynrelm}{
                type=abbreviationslist,
                name={PMSynRelM},
                description={Permanent Magnet Assisted Synchronous Reluctance Motor}
}
\newglossaryentry{abbreviation:synrelm}{
                type=abbreviationslist,
                name={SynRelM},
                description={Synchronous Reluctance Motor}
}
\newglossaryentry{abbreviation:pm}{
                type=abbreviationslist,
                name={PM},
                description={Permanent Magnets}
}
\newglossaryentry{abbreviation:pmsm}{
                type=abbreviationslist,
                name={PMSM},
                description={Permanent Magnet Synchronous Motor}
}
\newglossaryentry{abbreviation:dsp}{
                type=abbreviationslist,
                name={DSP},
                description={Digial Signal Processor}
}
\newglossaryentry{abbreviation:foc}{
                type=abbreviationslist,
                name={FOC},
                description={Field Oriented Control}
}
\newglossaryentry{abbreviation:dtc}{
                type=abbreviationslist,
                name={DTC},
                description={Direct Torque Control}
}

\newglossaryentry{symbol:Pn}{
    type=symbolslist, % glossary
    name=$P_\text{n}$, % jméno v seznamu
    description={jmenovitý výkon}, %popis
    symbol = (W),
    sort=P % seředit podle
}


\newglossarystyle{myStyleAbbreviations}{
\renewenvironment{theglossary}%
     {\begin{longtable}[l]{llp{\glsdescwidth}p{\glspagelistwidth}}}%
     {\end{longtable}}%
  \renewcommand*{\glossaryheader}{}%
  \renewcommand*{\glsgroupheading}[1]{}%
  \renewcommand{\glossentry}[2]{%
  \glsentryitem{##1} \glstarget{##1}{##2} &
    \textbf{\glossentryname{##1}} &
    \glossentrydesc{##1} &
    ##2\tabularnewline
  }%
  \renewcommand*{\glsgroupskip}{}%  Pokud chci seskupovat podle abeced: \renewcommand*{\glsgroupskip}{ & \\}
}


\newglossarystyle{myStyleSymbols}{
  \renewenvironment{theglossary}%
    {\begin{longtable}[l]{llp{\glsdescwidth}p{\glspagelistwidth}}}%
    {\end{longtable}}%
  \renewcommand*{\glossaryheader}{}%
  \renewcommand*{\glsgroupheading}[1]{}%
  \renewcommand{\glossentry}[2]{%
    \glsentryitem{##1} \glstarget{##1}{\glossentryname{##1}} &
    \glossentrysymbol{##1} &
    \glossentrydesc{##1} &
    ##2\tabularnewline
  }%
  \renewcommand{\subglossentry}[3]{%
     &
     \glssubentryitem{##2}%
     \glossentrysymbol{##2} &
     \glstarget{##2}{\strut}\glossentrydesc{##2} & ##3\tabularnewline
  }%
  \renewcommand*{\glsgroupskip}{%
   }% Pokud chci seskupovat podel abecedy  \ifglsnogroupskip\else & & &\tabularnewline\fi
}
\renewcommand{\glossarypreamble}{\vspace*{-\baselineskip}} % deleting line after glossaries title


%% END OF GLOSSARIES %%

% this works with LuaLaTex and fontspec package %
\DeclareCaptionFont{times}{\fontspec{Times New Roman Italic}}

% labelfont and textfont defined here only works with previous declarecaptionfont times and fontspec
\captionsetup{labelfont=times, textfont=times, labelsep=space}%no separator in captions


%\bibliographystyle{czechiso} %czechiso.bst in folder is needed for this style to work, available at http://www.fit.vutbr.cz/~martinek/latex/czechiso.html%


\usepackage{chngcntr} % for numbered figures with sections
\usepackage{tocloft} % better TOC

%\usepackage{a4wide}%širší a4%
\usepackage[inner=3cm,outer=2cm,top=2.5cm,bottom=2.5cm,footskip=1cm]{geometry}%for propper margins
\usepackage{textcase} % for making text uppercase without caps \MakeTextUppercase
 
 
\usepackage{titlesec} % for spacing text after sections
\usepackage{parskip}[] % for working \parskip
\newcommand{\sectionbreak}{\clearpage} % maybe for SECTIONS on a new page

\usepackage[titletoc]{appendix} % For appendix - přílohy, titletoc is crucial
%\renewcommand{\appendixname}{Příloha}

\setlength{\parindent}{0.5cm} % setting indent of paragraph to 0.5cm
\setlength{\parskip}{0em} % setting parskip to 0 for \titleformat to work properly with parskip package

\usepackage{colortbl} % for colored cells in tables
\usepackage{xcolor} % for color definitions

%% COLOR DEFINITIONS %%
\definecolor{ctublue}{HTML}{0065BD} % defining ctu color
\definecolor{ctugreen}{HTML}{A2AD00}
\definecolor{ctured}{HTML}{C60C30}
\definecolor{ctuyellow}{HTML}{F0AB00}
\definecolor{ctugreenyblue}{HTML}{00B2A9}
\definecolor{ctulightblue}{HTML}{6AADE4}
\definecolor{ctuorange}{HTML}{E05206}
\definecolor{lightgray}{HTML}{D1D5DB}
\definecolor{codeblue}{HTML}{D9E2F3}
\definecolor{codegreen}{rgb}{0,0.6,0}
\definecolor{codegray}{rgb}{0.5,0.5,0.5}
\definecolor{codepurple}{rgb}{0.58,0,0.82}
\definecolor{backcolour}{rgb}{0.95,0.95,0.92}

% defining spacing of titles
\titlespacing*{\section}{0em}{1em}{-\parskip} % spacing text after sections from titlesec package
\titlespacing*{\subsection}{0em}{1em}{-\parskip} % spacing text after sections from titlesec package
\titlespacing*{\subsubsection}{0em}{1em}{-\parskip} % spacing text after sections from titlesec package

% defining style for titles
% when you want section/sub/subsub to be black, delete \color{ctublue}
\titleformat{\section}{\color{ctublue}\fontspec{Times New Roman}\fontsize{15}{15}\bfseries}{\thesection}{2.1em}{}%defining title sizes by word template
\titleformat{\subsection}{\color{ctublue}\fontspec{Times New Roman}\fontsize{14}{14}\bfseries}{\thesubsection}{1.53em}{}%defining title sizes by word template
\titleformat{\subsubsection}{\color{ctublue}\fontspec{Times New Roman}\fontsize{13}{13}\bfseries}{\thesubsubsection}{1em}{}%defining title sizes by word template


\usepackage{ctable} % imports xtable with booktabs
\usepackage{multicol} % for multicolumn feature in tables

\usepackage{listings} % for code environments - \begin{lstlisting}

% solving problems with ) literal to be coded in lstlisting as it should be
\makeatletter
\patchcmd{\lsthk@SelectCharTable}{)}{`}{}{} 
\makeatother 

\lstdefinestyle{zakopal}{
    backgroundcolor=\color{codeblue},   
    commentstyle=\color{codegray},
    keywordstyle=\color{ctured},
    numberstyle=\tiny\color{codegray},
    stringstyle=\color{ctuorange},
    basicstyle=\ttfamily\small,
    breakatwhitespace=false,         
    breaklines=true,                 
    captionpos=b,                    
    keepspaces=true,                 
    numbers=left,                    
    numbersep=5pt,                  
    showspaces=false,                
    showstringspaces=false,
    showtabs=false,                  
    tabsize=2
}
\lstset{style=zakopal}
\renewcommand{\lstlistingname}{Code}% renaming Listing -> Kód 
\renewcommand{\lstlistlistingname}{List of codes}% renaming List of Listings -> Seznam kódů

\lstdefinelanguage{SCL}
{morekeywords={FUNCTION_BLOCK,BEGIN,NOT,END_FUNCTION_BLOCK,FUNCTION,VOID,VAR_INPUT,END_VAR,VAR_IN_OUT,IF,
THEN,END_IF,END_FUNCTION,BOOL,FALSE,TRUE},
sensitive=false,
morecomment=[l]{//},
morestring=[b]",
literate={;}{{\textcolor{ctuorange}{;}}}{1}
{:}{{\textcolor{ctuorange}{:}}}{1}
{)}{{\textcolor{ctuorange}{)}}}{1}
{(}{{\textcolor{ctuorange}{(}}}{1}
{=}{{\textcolor{ctuorange}{=}}}{1}
{,}{{\textcolor{ctuorange}{,}}}{1},} %basic SCL language for siemens defined%

\lstdefinelanguage{xdc}
{morekeywords={set_property, current_design, get_ports},
sensitive=false,
morecomment=[l]{\#}} %basic xdc file in Vivado syntax highlighting%

\lstdefinelanguage{xsct}
{morekeywords={xsct, hsi, open_hw_design, -createdts, -hw, -zocl, -platform-name, -overlay, -compile, -out, exit, -git-branch},
alsoletter={-},
sensitive=false,
morecomment=[l]{\#}} %xsct (Xilinx Software Command-Line Tools)%


\lstdefinelanguage{Text}
{morekeywords={},
alsoletter={-},
sensitive=false,
morecomment=[l]{//},
morecomment=[l]{\#}} %basic text%

\lstdefinelanguage{devicetree}
{morekeywords={chosen, bootargs, stdout-path, compatible, status},
alsoletter={-},
stringstyle=\color{ctuorange},
moredelim=[s][\color{ctuorange}]{"}{"},
sensitive=false,
morecomment=[l]{\#},
literate={\{}{{\textcolor{ctured}{\{}}}{1}
{\}}{{\textcolor{ctured}{\}}}}{1}
} %devicetree%

\lstdefinelanguage{json}
{morekeywords={},
upquote=true,
morestring=[b]",
stringstyle=\color{ctuorange},
moredelim=[s][\color{ctuorange}]{"}{"},
sensitive=false,
morecomment=[l]{\#},
literate=
     *{0}{{{\color{ctured}0}}}{1}
      {1}{{{\color{ctured}1}}}{1}
      {2}{{{\color{ctured}2}}}{1}
      {3}{{{\color{ctured}3}}}{1}
      {4}{{{\color{ctured}4}}}{1}
      {5}{{{\color{ctured}5}}}{1}
      {6}{{{\color{ctured}6}}}{1}
      {7}{{{\color{ctured}7}}}{1}
      {8}{{{\color{ctured}8}}}{1}
      {9}{{{\color{ctured}9}}}{1}
      {\{}{{{\color{ctured}{\{}}}}{1}
      {\}}{{{\color{ctured}{\}}}}}{1}
      {[}{{{\color{ctured}{[}}}}{1}
      {]}{{{\color{ctured}{]}}}}{1},
} %json%

\lstdefinelanguage{pseudocode}
{morekeywords={if, else, positive, negative, edge, end},
alsoletter={-},
stringstyle=\color{ctuorange},
moredelim=[s][\color{ctuorange}]{"}{"},
sensitive=false,
morecomment=[l]{\//},
literate={\{}{{\textcolor{ctured}{\{}}}{1}
{\}}{{\textcolor{ctured}{\}}}}{1}
{(}{{{\color{ctured}{(}}}}{1}
{)}{{{\color{ctured}{)}}}}{1}
} %devicetree%


%%change in previous commands 2.1 em , 1.53em and 1em to 1em to be easy indented not the same
\begin{document}
\fontspec{Times New Roman}

\counterwithin{figure}{section}%changing counter of figure, at each section the numbering resets
\counterwithin{table}{section}%changing counter of table, at each section the numbering resets
\counterwithin{equation}{section}%changing counter of equation, at each section the numbering resets

\counterwithin{lstlisting}{section}%counter of lstlist - codes, reseting at each section%


\renewcommand{\thefigure}{\thesection~-~\arabic{figure}}%defining style of countering
\renewcommand{\thetable}{\thesection~-~\arabic{table}}
\renewcommand{\theequation}{\thesection~-~\arabic{equation}}
\renewcommand{\thelstlisting}{\thesection~-~\arabic{lstlisting}}%delfining style for lstlisting codes, needs to be after begin document as previous renewcommand%

\renewcommand*{\cftsecdotsep}{1}  % use dots in the section entries and their step
\renewcommand*{\cftsubsecdotsep}{1}
\renewcommand*{\cftsubsubsecdotsep}{1}
\renewcommand*{\cftsecnumwidth}{4em} % increase space for Roman numerals
\renewcommand*{\cftsubsecnumwidth}{4em} %numbering width
\renewcommand*{\cftsubsubsecnumwidth}{4em} %numbering width
\renewcommand*{\cftsubsubsecindent}{0em}%no indent for subsubsection
\renewcommand*{\cftsubsecindent}{0em}%no indent for subsection
\renewcommand*{\cftsecindent}{0em}%no indent for subsection



\renewcommand*{\cftfigdotsep}{1}  % use dots in the figure entries and their step
\renewcommand*{\cftfignumwidth}{4em}
\renewcommand*{\cftfigindent}{0em}

\renewcommand*{\cfttabdotsep}{1}  % use dots in the figure entries and their step
\renewcommand*{\cfttabnumwidth}{4em}
\renewcommand*{\cfttabindent}{0em}

\renewcommand{\cftsecfont}{\fontspec{Times New Roman}\large \bfseries}
\renewcommand{\cftsubsecfont}{\fontspec{Times New Roman}}
\renewcommand{\cftsubsubsecfont}{\fontspec{Times New Roman}}

\renewcommand{\cftfigfont}{\fontspec{Times New Roman}}
\renewcommand{\cfttabfont}{\fontspec{Times New Roman}}

\renewcommand*\contentsname{\textcolor{ctublue}{\MakeTextUppercase{\fontspec{Times New Roman}Table of Contents}}}
\renewcommand{\listtablename}{{\fontspec{Times New Roman}\textcolor{ctublue}{\MakeTextUppercase{{List of tables}}}}}
\renewcommand{\listfigurename}{{\fontspec{Times New Roman}\textcolor{ctublue}{\MakeTextUppercase{{List of figures}}}}}


%% TITLE PAGE %%
\setcounter{figure}{0}

\begin{titlepage}
	\begin{center}

\begin{figure}[H]
	\begin{center}
		\includegraphics[scale=1]{src/misc/symbol_cvut_konturova_verze.pdf}
	\end{center}
\end{figure}
	{\Large{\textbf{CZECH TECHNICAL UNIVERSITY IN PRAGUE}}}\\
	{\textbf{Faculty of Electrical Engineering}}\\
	{\textbf{Department of Electric Drives and Traction}}
	
	\vspace{3cm}
	
	
	{\Large\textbf{A Brief Report on Permanent Magnet Assisted Synchronous Reluctance Motors}}
	
	\vspace{1cm}
	
	%{\Large\textbf{Possibilities of Using SoC Platform Processors for Controlling Electric Drives}}
	
	%\vspace{2cm}
	
	
	\end{center}
	
	\vspace{3cm}
	
	%\noindent Studijní program: Elektrotechnika, Energetika a Management\\
	%\noindent Studijní obor: Elektrické pohony
	
	\vspace{0.5cm}
	%\noindent Vedoucí práce: doc. Ing. Jan Bauer, Ph.D.
	
	\vfill
	
\begin{center}

	\large{\textbf{Petr Zakopal}}\\
	\large{\textbf{Prague 2023}}
	\end{center}
\end{titlepage}


\newpage
\pagenumbering{gobble} %disabling page numbering

\newpage


%%ZADÁNÍ PRÁCE
%verze pro TISK - jen s NEW PAGE


%ONLINE VERZE - se zadáním BEZ PODPISŮ
% online verze - odkomentovat následující dva řádky
%\null\newpage
%\includepdf[]{src/docs/zadani_bez_podpisu.pdf}

%\newpage
%\cleardoublepage
\null\newpage

\pagenumbering{Roman}
\setcounter{page}{2}%%3 NUTNO řešit dle zadání etc.

%% PROHLASENI A PODEKOVANI %%
%  \noindent \textcolor{ctublue}{{\Large{\textbf{\MakeTextUppercase{Prohlášení}}}}}\\
 			Prohlašuji, že jsem předloženou práci vypracoval samostatně a že jsem uvedl veškeré použité informační zdroje v~souladu s~Metodickým pokynem o~dodržování etických principů při přípravě vysokoškolských závěrečných prací.\\
 		\vspace{1.5cm}
		
	

 	\noindent	V~Praze dne \rule{3.5cm}{0.4pt} \hspace{6.6cm}  \rule{4cm}{0.4pt}
	
 	\hspace{12.65cm}Petr Zakopal


 		\vspace{14cm}
		
 	\noindent	\textcolor{ctublue}{{\Large{\textbf{\MakeTextUppercase{Poděkování}}}}}\\
 	Tímto bych rád poděkoval vedoucímu této práce doc. Ing. Janu Bauerovi, Ph.D. za skvělé vedení práce a cenné rady při jejím vytváření. Dále bych rád poděkoval všem, kteří mě v~mém dosavadním studiu podporovali.
	


%%ABSTRAKT%%
%  \newpage
 %\addcontentsline{toc}{section}{3\quad Abstrakt a klíčová slova}%Added citations to TOC%
 %\begin{comment}
 \begin{minipage}[t]{7.37cm}
 		\raggedright
 	\textcolor{ctublue}{\Large{\textbf{\MakeTextUppercase{Abstrakt}}}}\\
    Textík.
	
 \end{minipage}%
 \hfill% --- important, otherwise it wont be so nice
 \begin{minipage}[t]{7.37cm}
 		\textcolor{ctublue}{\Large{\textbf{\MakeTextUppercase{Abstract}}}}\\
	Textík v aj.	
 \end{minipage}
% %\end{comment}
% 	%\textcolor{ctublue}{\Large{\textbf{\MakeTextUppercase{Abstrakt}}}}\\

% 	%\textcolor{ctublue}{\Large{\textbf{\MakeTextUppercase{Abstract}}}}\\

% \newpage



%% TABLE OF CONTENTS %%
\tableofcontents
\newpage%
\flushbottom % vyčištění stránky
\newpage
\vspace{0pt}

%% LIST OF FIGURES %%
\listoffigures % seznam obrázků
\flushbottom % vyčištění stránky
\newpage

%% LIST OF TABLES %%
\listoftables
\flushbottom

\newpage


%% BLANK PAGE AFTER TOC, TOF, TOT %%
\pagenumbering{gobble}
\null\newpage

% setting the page numbering to start from 1, where the main text starts
\setcounter{page}{1}
\pagenumbering{arabic}

\section{Introduction}
\gls{symbol:Pn} \gls{abbreviation:asm}

\section{Design}
The Permanent Magnet assisted synchronous reluctance motor (\gls{abbreviation:pmsynrelm}) is widely used for it's significant advantages of small size, low loss, high efficiency, better performance than plain synchronous reluctance motors \gls{abbreviation:synrelm} and wide constant power to speed range. \cite{xinmin-design-of-permanent-magnet-assisted-synch-rel-m-with-low-torque-ripple,huynh-design-and-analysis-of-perm-as-synch-rel-m}
    \subsection{Stator and Rotor}
        There are many solutions on how to connect the stator winding. Research has been carried out for standard Delta or Start winding, but to increase the torque for same stator current the combined Start-Delta winding was proposed. The first research has been caried out for standard \gls{abbreviation:synrelm} in \cite{ibrahim-an-improved-torque-density-synchronous-reluctance-machine-with-a-combined-star-delta-winding-layout} and then extended to \gls{abbreviation:pmsynrelm} prototypes in \cite{ibrahim-permanent-magnet-assisted-synchronous-reluctance-motor-employing-a-hybrid-star-delta-winding-for-high-speed-applicaitons}. The main idea of the hybrid Delta-Star connection is to split the standard phase wiring into two parts. The one part is for the Delta connection, the other for Star connection. Then the coils of wiring are connnected to series. Motors utilizing hybrid stator winding with \gls{abbreviation:pm}s inserted in the rotor flux bariers exhibit constant power factor over 0.9

        \begin{figure}[htbp!]
            \centering
            \includegraphics[width=0.5\textwidth]{src/png/hybrid-star-delta-wiring.png}
            \caption{Hybrid Star-Delta wiring of \gls{abbreviation:pmsynrelm}. {\textcolor{ctured}{CHANGE THIS IMAGE FOR YOUR OWN, IT IS FROM \cite{ibrahim-permanent-magnet-assisted-synchronous-reluctance-motor-employing-a-hybrid-star-delta-winding-for-high-speed-applicaitons}}}}
            \label{fig:hybrid-star-delta-wiring}
        \end{figure}

        In \cite{ibrahim-permanent-magnet-assisted-synchronous-reluctance-motor-employing-a-hybrid-star-delta-winding-for-high-speed-applicaitons} the authors manufactured proposed four prototypes. The prototypes consist of two stators, with either conventional star winding or hybrid star-delta winding, and two rotors, with ferrite permanent magnets or without. Maxwell transient simulations were carried out on the four prototypes, which were then manufactured and experiment using the simulation results was conducted.
\par
    According to \cite{ibrahim-permanent-magnet-assisted-synchronous-reluctance-motor-employing-a-hybrid-star-delta-winding-for-high-speed-applicaitons} the researches state, that when using the hybrid stator winding connection, the efficiency increase is rather low compared to efficiency increase when comparing \gls{abbreviation:synrelm} with and without \gls{abbreviation:pm}s.\par
    The design of the \gls{abbreviation:pmsynrelm} rotor with \gls{abbreviation:pm} oriented solely in the $q$-axis is depicted in the figure \ref{fig:pmsynrelm-rotor-magnets-q-axis}.
    
    \begin{figure}[htbp!]
            \centering
            \includegraphics[width=0.5\textwidth]{src/png/pmsynrelm-rotor-magnets-q-axis.png}
            \caption{Rotor design of a Permanent Magnet Assisted Synchronous Reluctance Motor with permanent magnets oriented solely in the $q$-axis. \cite{tavernini-design-and-optimisation-of-energy-efficient-pmsynrelm-for-electric-vehicles}}
            \label{fig:pmsynrelm-rotor-magnets-q-axis}
    \end{figure}



    \subsection{Magnets}
        \gls{abbreviation:pmsynrelm} are very often compared to Permanent Magnet Synchronous Motors (\gls{abbreviation:pmsm}) used in the automotive field in terms of power and torque density, efficiency and costs. Though the \gls{abbreviation:pmsm} are very popular \cite{morimoto-experimental-evaulation-of-a-rare-earth-free-pmasynrm-with-ferrite-magnets-for-automotive-applications}, the \gls{abbreviation:pm}s used in their design often consist of rare-earth materials such as neodymium or dysprosium. That is the reason why \gls{abbreviation:pmsynrelm} motors with rare-earth-free materials are now being the subject of many research studies. Experiments comparing the production-used \gls{abbreviation:pmsm} and experimental prototype \gls{abbreviation:pmsynrelm} show, that the proposed prototype in \cite{mashiro-performance-of-mpasynrm-with-ferrite-magnets-for-ev-hv-applications-considering-productivity} achieve close values of power density and an efficiency as rare-earth \gls{abbreviation:pmsm} counterpart, but with much lower costs \cite{haiwei-low-cost-ferrite-pm-assisted-synchronous-reluctance-machine-for-electric-vehicles}.\par
It has been observed, that when inserting the \gls{abbreviation:pm} in the center of the flux barrier, a magnetic flux lines are forced to pass through the flex barriers in the $q$-axis. This results in the decreased linked magnetic flux in the $q$-axis and therefore improves the output torque. \cite{ibrahim-permanent-magnet-assisted-synchronous-reluctance-motor-employing-a-hybrid-star-delta-winding-for-high-speed-applicaitons, ngo-performance-analysis-of-synchronous-reluctance-motor-with-limited-amount-of-permanent-magnet} The two general types of rotor with embedded \gls{abbreviation:pm} are depicted in figure \ref{fig:pmsynrelm-rotor-magnets-position}.


    \begin{figure}[htbp!]
            \centering
            \includegraphics[width=0.8\textwidth]{src/png/pmsynrelm-rotor-magnets-position.png}
            \caption{Different approaches to permanent magnet orientation in the rotor of a Permanent Magnet Assisted Synchronous Reluctance Motor. (\textbf{a}) PM embedded along the flux bariers, facing the $q$-axis; (\textbf{b}) Permanent magnets are crossing the flux bariers, therefore facing the $d$-axis. \cite{ngo-performance-analysis-of-synchronous-reluctance-motor-with-limited-amount-of-permanent-magnet}}
            \label{fig:pmsynrelm-rotor-magnets-position}
    \end{figure}


\section{Control}

    \subsection{Mathematical model}
        The stator voltage equation of \gls{abbreviation:pmsynrelm} denoted in the general axis $k$ is as follows

        \begin{equation}
            \underline{u}^k_1 = \text{R}_\text{s} \underline{i}^k_1 + \frac{\dd \underline{\psi}^k_1 }{\dd t} + j \omega_k \underline{\psi}^k_1.
        \end{equation}

        Where $\underline{u}^k_1$ (V) is space vector of stator voltage, $R_\text{s}$ ($\Omega$) is stator rezistance, $\underline{i}^k_1$ (A) space vector of a stator current, $\underline{\psi}^k_1$ (Wb) space vector of a stator flux linkeage, $\omega_k$ (rad $\text{s}^{-1}$) general angular speed.\par
        The voltage equation denoted in $dq$-axis is as follows

        \begin{equation}
            \underline{u}^{dq}_1 = \text{R}_\text{s} \underline{i}^{dq}_1 + \frac{\dd \underline{\psi}^{dq}_1 }{\dd t} + j \omega_1 \underline{\psi}^k_1,
        \end{equation}
        
        where $\omega_1$ (rad $\text{s}^{-1}$) is electrical angular speed of a stator rotating magnetic field. When the equation is denoted in vector components and the subscript "1" for stator is omitted and the axis are newly denoted by the variables subscript

        \begin{equation}
            \underline{u}_d = \text{R}_\text{s} i_\text{d} + \frac{\dd \psi_d}{\dd t} - \omega_1\psi_q,
        \end{equation}
        \begin{equation}
            \underline{u}_q = \text{R}_\text{s} i_\text{q} + \frac{\dd \psi_q}{\dd t} + \omega_1\psi_d.
        \end{equation}

        Equations for flux linkeages denoted in the $d,q$-axis, when \gls{abbreviation:pm}s are embedded along the $q$-axis are

        \begin{equation}\label{eq:d-axis-flux-linkeage}
            \psi_d = \text{L}_d i_\text{d},
        \end{equation}

        \begin{equation}\label{eq:q-axis-flux-linkeage-general}
            \psi_q = \text{L}_q i_\text{q} + \psi_\text{PM}.
        \end{equation}

        Where $\text{L}_q$ (H), $\text{L}_d$ (H) are inductances in $d$-axis and $q$-axis respectively, $\psi_\text{PM}$ (Wb) is a flux linkeage of permanent magnets. Very often the \gls{abbreviation:pm} flux linkeage is oriented negatively in the $q$-axis when respecting the vector orientation the equation \ref{eq:q-axis-flux-linkeage-general} can be rewritten as

        \begin{equation}\label{eq:q-axis-flux-linkeage-rewritten}
            \psi_q = \text{L}_q i_\text{q} - \psi_\text{PM}.
        \end{equation}
        
        The general equation for electromagentic torque is then

        \begin{equation}\label{eq:torque-general}
            \begin{gathered}
                T = \frac{3}{2} \text{p}_\text{p} | \underline{\psi_{dq}} \times \underline{i_{dq}} | = \frac{3}{2} \text{p}_\text{p} (\psi_d i_q - \psi_q i_d).
            \end{gathered}
        \end{equation}

        where $\text{p}_\text{p}$ (-) is number of pole pairs.\par

        After the substituion of \ref{eq:d-axis-flux-linkeage} and \ref{eq:q-axis-flux-linkeage-rewritten} to \ref{eq:torque-general} the torque euation may be rewritten


        \begin{equation}\label{eq:torque-pmsynrelm}
            \begin{gathered}
                M = \frac{3}{2} \text{p}_\text{p} (\text{L}_d i_d i_q - (\text{L}_q i_q -\psi_\text{PM}) i_d) = \frac{3}{2} \text{p}_\text{p} (\text{L}_d i_d i_q - \text{L}_q i_q i_d + \psi_\text{PM} i_d).
            \end{gathered}
        \end{equation}

        \par
        As can be seen from eq. \ref{eq:torque-general}, when the linkeage flux of \gls{abbreviation:pm}s is oriented negatively to the $q$-axis (as presented), higher value of flux linkeages make the electromagnetic torque higher.\par
        
        Graphical expression of the phasor diagram for the \gls{abbreviation:synrelm} is depicted in the figure \ref{fig:cad-pmasynrm-vector-diagram}.
        
        \begin{figure}[htbp!]
            \centering
            \includegraphics[width=0.70\textwidth]{src/pdf/cad-pmasynrm-vector-diagram.pdf}
            \caption{Phasor diagram for the \gls{abbreviation:synrelm} when the flux of permanent magnets $\psi_\text{m}$ is oriented in the negative $q$-axis direction.}
            \label{fig:cad-pmasynrm-vector-diagram}
        \end{figure}
    
    \par
    Effect of different layers of \gls{abbreviation:pm} in rotor on realized the phasor diagram may be observed in \cite{huynh-design-and-analysis-of-perm-as-synch-rel-m}.

    \subsection{Control strategies}

        There are many options on how to control the \gls{abbreviation:pmsynrelm}. The principles may be divided in two major groups: \textbf{Scalar Control} and \textbf{Vector Control}. The two major subcategories of vector control strategies are \textbf{Field Oriented Control} (\gls{abbreviation:foc}) and \textbf{Direct Torque Control} (\gls{abbreviation:dtc}). These strategies then may use different aproaches to achieve the desired results of less torque ripple and dynamic performance. \cite{dwivedi-review-on-control-strategies-of-permanent-magnet-assisted-synchronous-reluctance-motor-drive} The general group decopmosition is depicted in the figure \ref{fig:pmsynrelm-control-strategies}.

        \begin{figure}[htbp!]
            \centering
            \includegraphics[width=0.70\textwidth]{src/png/pmsynrelm-control-strategies.png}
            \caption{General diagram depicting major groups of control strategies for \gls{abbreviation:pmsynrelm}. Graph inspired by \cite{dwivedi-review-on-control-strategies-of-permanent-magnet-assisted-synchronous-reluctance-motor-drive}}
            \label{fig:pmsynrelm-control-strategies}
        \end{figure}

        \subsubsection{Scalar Control}
            Scalar drive control is straighforward solution for controlling the drive. It is relatively simple to execute and there is no need of a high performance Digital Signal Processors (\gls{abbreviation:dsp}s). However Scalar Control can not provide the dynamic performance and speed control compared to \gls{abbreviation:foc} and \gls{abbreviation:dtc}. \cite{dwivedi-review-on-control-strategies-of-permanent-magnet-assisted-synchronous-reluctance-motor-drive}\par
            Scalar control is mainly known as a $V$/$f$ (Voltage/frequency) control. The control methods mainly produce output voltage so the ratio between voltage and frequency is kept constant for the magnetizing flux to be the highest possible, so the torque is possibly maximized as well. Another methods use $I$/$f$ (current/frequency) control based strategies. \cite{heidari-a-review-of-synchronour-relucatence-motor-drive-advancements}

        \subsubsection{Vector Control}
            Vector control strategies became increasingly popular due to the lower cost and higher computational power of available \gls{abbreviation:dsp}s. \cite{dwivedi-review-on-control-strategies-of-permanent-magnet-assisted-synchronous-reluctance-motor-drive}
            \par
            \gls{abbreviation:foc} control mainly uses the theory of space vectors and \gls{abbreviation:dtc} the theory of controlling the electromagnetic torque and magnetic flux based on the desired speed and magnetization. Control strategies are different but objective is the same. The main aim of vector control strategies is to achieve the desired torque and flux values based on the reference values which are set as an input to the control strategy. \cite{heidari-a-review-of-synchronour-relucatence-motor-drive-advancements, dwivedi-review-on-control-strategies-of-permanent-magnet-assisted-synchronous-reluctance-motor-drive}
            
            \vspace*{1.5cm}
             \hspace*{-\parindent} \textbf{Maximum Torque Per Ampere (\gls{abbreviation:mtpa}) \gls{abbreviation:pmsynrelm} control}\par
                \parskip The main objective of the \gls{abbreviation:mtpa} strategy is to achieve the reference (desired) torque with minimum value of stator currents in $d$ and $q$-axis ($i_d$ and $i_q$). According to \cite{dwivedi-review-on-control-strategies-of-permanent-magnet-assisted-synchronous-reluctance-motor-drive}, there are multiple methods how to realize the \gls{abbreviation:mtpa}.\par
                The control strategy is parameter dependent. In \cite{niazi-robust-maximum-torque-per-ampere-control-of-pmsynrelm} authors present a robust online parameter estimation technique which improves the control strategy. With calculated and estimated parameters and measured stator currents the torque, which would the machine provide is calculated and then used as a reference value for further calculations. The proposed controller provides increased robustness againts the variations of motor parameters.

            \vspace*{1.5cm}
             \hspace*{-\parindent} \textbf{Maximum Torque Per Voltage (\gls{abbreviation:mtpv}) \gls{abbreviation:pmsynrelm} control}\par
                \parskip The higher the speed of rotor, the larger the magnitude of the back electromotive force (\gls{abbreviation:emf}), the larger magnitude of voltage provided from source is needed. When speed reaches the value, where nominal source voltage is reached, then the current flowing through stator wires must decrease due to the back \gls{abbreviation:emf}. Thus the voltage value restricts the current based on the rotor speed. In \cite{sanz-analitical-maximum-torque-per-volt-control-strategu-of-an-interior-permanent-magnet-synchronous-motor-with-very-low-battery-voltage} the exemplar curves presenting \gls{abbreviation:mtpv} are depicted. Another mathematical expression of the \gls{abbreviation:mtpv} trajectoriy is presented in \cite{fletcher-operation-along-the-maximum-torque-per-voltage-trajectory-in-a-direct-torque-and-flux-controlled-interior-permament-magnet-synchronous-motor}. Cited paper also depicts the exemplar trajectories in the $i_d$-$i_q$ plane.
                \par
                The \gls{abbreviation:mtpv} trajectories are plotted based on the current values in $i_d$-$i_q$, where the possible torque is at the peak. The maximum torque value in the diagram is based on the operation speed.


            \vspace*{1.5cm}
             \hspace*{-\parindent} \textbf{Unity Power Factor Control (\gls{abbreviation:upfc})}\par
             \parskip In wide variety of applications it is required for the machine to work with maximum power factor. It is preferable to achieve a unity power factor, to eliminate reactive power consumption. In \cite{moussa-unity-power-factor-control-of-permanent-magnet-motor-drive-system} two methods are proposed: \textit{1) controling the $d$-axis stator current $i_d$} and \textit{2) controling the angle of stator current space vector $i_\text{stator}$}.\par
             According to \cite{moussa-unity-power-factor-control-of-permanent-magnet-motor-drive-system} the \gls{abbreviation:upfc} allows wider speed range where the torque is constant. This results in a higher output power of the drive.\par

             \vspace*{1cm}
             \textit{1) controling the $d$-axis stator current $i_d$}\par
             \parskip Method compares the space angles of stator current and voltage space vectors to achieve the unity power factor. From the unity power factor condition the current $i_d$ may then be expressed. Using the expressed current value, the voltage equations of the machine then may be modified to evaulate the steady-state performance at a unity power-factor. \cite{moussa-unity-power-factor-control-of-permanent-magnet-motor-drive-system}

             \vspace*{1cm}
             \textit{2) controling the angle of stator current space vector $i_\text{stator}$}\par
             \parskip This method forces the space vector of a stator current to be aligned with space vector of a stator electromagnetic force space vector. When the space vectors of a stator voltage and current will coincide the unity power factor will be achieved. \cite{moussa-unity-power-factor-control-of-permanent-magnet-motor-drive-system}

\section{Comparation to others}

\section{Usage}

%% Conclusion %%
\newpage
\addcontentsline{toc}{section}{\numberline{}Conclusion} 
\section*{Conclusion}

\flushbottom % vyčištění stránky

% konec závěru

\newpage
\setmonofont{Times New Roman}

%% REFERENCES %%
\printbibliography[title={{References}}]	
\nocite{*}
\setmonofont{CourierPrime-Regular}
\addcontentsline{toc}{section}{\numberline{}References} % Adding citations to TOC %

%% APPENDIX %%
\appendix
\titleformat{\section}{\color{ctublue}\fontspec{Times New Roman}\fontsize{15}{15}\bfseries}{Appendix \thesection:}{2.1em}{}

\begin{appendices}
	\section{List of symbols and abbreviations}

		\printglossary[type=abbreviationslist, style = myStyleAbbreviations]

		\fbar
		%\newpage
		\printglossary[type=symbolslist, style =  myStyleSymbols]

	\end{appendices}
\end{document}
